% =======================================================
% ============== Preamble ==================
% =======================================================
\documentclass[12pt,a4paper]{article} %Optional = twoside

\usepackage[english]{babel}
\usepackage[utf8]{inputenc}
\usepackage[T1]{fontenc}
\usepackage{amsmath}
\usepackage{amssymb}
\usepackage{gauss}
\usepackage{physics}
\usepackage{siunitx} %\si{} giver ordentlige enheder
\usepackage{caption}			%Tillader costum figurtekster
\usepackage[pdftex]{graphicx}	% Gør det muligt at inkludere grafik (bl.a. png og jpg) i dokumentet.
\usepackage{lastpage}			% Gør det muligt at skrive sidetallet af den sidste side.
\usepackage[textfont={rm, it}, labelfont={bf}]{caption}	% Definer hvordan figur tekster ser ud.
\usepackage{gensymb}			% Tilføjer ekstra symbol som grader celsius.
% Aktiverer todonotes pakken
\usepackage[bordercolor=gray!20,backgroundcolor=blue!10,textsize=footnotesize]{todonotes}
%\setlength{\marginparwidth}{0.6cm}	%Fix til brud ved twoside

\usepackage{icomma}				% Gør brug af komma som decimalseperator mulig.
\usepackage{pdfpages}			%Gør det muligt at indsætte PDF
\usepackage{cancel}
\usepackage{float}
\usepackage{slashed}
\usepackage{listings,xcolor}
\usepackage{setspace}
\usepackage{feynmp-auto}
\usepackage{hyperref}
\usepackage{neuralnetwork}
\usepackage{pgfplots}
\usepackage{subcaption}
\pgfplotsset{compat=1.11} % <-- added

% ============== Setting up listings =========================
\definecolor{codegreen}{rgb}{0,0.6,0}
\definecolor{codegray}{rgb}{0.5,0.5,0.5}
\definecolor{codepurple}{rgb}{0.58,0,0.82}
\definecolor{backcolour}{rgb}{0.95,0.95,0.92}
\lstdefinestyle{mystyle}{
	backgroundcolor=\color{backcolour},   
	commentstyle=\color{codegreen},
	keywordstyle=\color{magenta},
	numberstyle=\tiny\color{codegray},
	stringstyle=\color{codepurple},
	basicstyle=\ttfamily\footnotesize,
	breakatwhitespace=false,         
	breaklines=true,                 
	captionpos=b,                    
	keepspaces=true,                 
	numbers=left,                    
	numbersep=5pt,                  
	showspaces=false,                
	showstringspaces=false,
	showtabs=false,                  
	tabsize=2
}
\lstset{style=mystyle}

% ============== Setting up tikz =========================

\usepackage{tikz}
\usetikzlibrary{calc,positioning,shadows.blur,decorations.pathreplacing,matrix,backgrounds,arrows.meta,chains}
\usepackage{etoolbox}
\usetikzlibrary{decorations.pathreplacing}
\usetikzlibrary{fadings}

% For convolution
\newcommand\numRowsK{3}
\newcommand\numColsK{3}
\newcommand{\K}[2]{% #1: row, #2: col
	\edef\Kcol##1##2##3{###2}%
	\edef\Krow##1##2##3{\noexpand\Kcol###1}%
	\Krow
	{1 0 1}
	{0 1 0}
	{1 0 1}%
}

% For maxpool
\newcommand*{\xMin}{0}%
\newcommand*{\xMax}{4}%
\newcommand*{\yMin}{0}%
\newcommand*{\yMax}{4}%

\newcommand*{\xMinR}{8}%
\newcommand*{\xMaxR}{10}%
\newcommand*{\yMinR}{1}%
\newcommand*{\yMaxR}{3}%

\definecolor{c1}{HTML}{9ACFC6}
\definecolor{c2}{HTML}{DABBD6}
\definecolor{c3}{HTML}{CBDCB9}
\definecolor{c4}{HTML}{9AC3E1}

\newcommand{\empt}[2]{$#1^{\langle #2 \rangle}$}
% ============== Setting up fancy headers ... =========================
\usepackage{fancyhdr}
\pagestyle{fancy}
\renewcommand{\sectionmark}[1]{\markright{\thesection.\ #1}}
\lhead{\korttitel}
\chead{\dato}
\rhead{\navn}
\fancyfoot[RE,LO]{\kursus}
\cfoot{}
\fancyfoot[LE,RO]{Page \thepage\ of \pageref{LastPage}}
\renewcommand{\headrulewidth}{0.5pt}
\renewcommand{\footrulewidth}{0.5pt}

\fancypagestyle{plain}{%
	\renewcommand{\headrulewidth}{0pt}%
	\fancyhf{}%
	\fancyfoot[L]{\kursus}%
	\fancyfoot[R]{Page \thepage\ of \pageref{LastPage}}%
}

% Øg teksthøjden med 2 cm på alle sider
\addtolength\textheight{2cm}
\addtolength\topmargin{-1cm}
\addtolength\marginparwidth{1.5cm}
\addtolength\headheight{1.6pt}

\makeindex

% ============== Various options =================================
\numberwithin{equation}{section}
\setlength{\jot}{2ex}

% ============== Shothand =================================
\newcommand{\gamfa}[1]{\frac{1}{\sqrt{1-\frac{#1^2}{c^2}}}}
\newcommand{\gamfain}[1]{{\sqrt{1-\frac{#1 ^2}{c^2}}}}
\newcommand{\emf}{\mathcal{E}}

\newcommand{\en}[1]{\ \text{#1}}
\newcommand{\paa}[1]{\left(#1\right)}
\newcommand{\pd}{\partial}


% ============== Dokument egenskaber =============================
\newcommand{\titel}{Advanced machine learning}
\newcommand{\korttitel}{Lecture Notes}
\newcommand{\dato}{\today}
\newcommand{\navn}{Mattias Ermakov Thing}
\newcommand{\kursus}{CP3 Masterclass}

\title{\titel}
\author{\navn}
\date{\dato}

\begin{document}
	% ==================================================================
	% ====== Hovedafsnit=================================================
	% ==================================================================
	
	\maketitle

\section{Introduction, GitHub, and neural networks}\label{sec:lecture1}
	The first lecture is focused on establishing prerequisites and leveling out the playing field such that everyone is on the same page as we dive into the details in the following lectures. We take a look at the structure of the course and GitHub where the course material is located. Then we will review the basics of neural network which will be the focus of this course.
	
	\subsection{Course overview}
		This course aim to introduce the students to various of methods of machine learning, primarily focused on neural networks and ways of training them. By the end of the course students should know about various types of neural network architectures and when to use a given architecture. We also look at ways to improve neural networks with physics guidance. The course consists of five two hour lectures with the following content:
		\begin{itemize}
			\item Introduction, GitHub, and neural networks
			\item Machine learning and methods of training
			\item Neural network architectures
			\item Physics guidance in neural networks
			\item TBD
		\end{itemize}
		Each lecture will come with relevant code and exercises to prepare for the next lesson. Although the exercises are not mandatory, it is highly recommended to do as it will improve practical skills. After the first lesson we will start the lectures by discussing the exercise from the previous lecture. The course material is available at: \url{https://github.com/CP3-Origins/advanced-machine-learning}. Beyond the lecture notes Google is your friend.
	
	\subsection{Prerequisites}
		This course will use Python and basic programming skills in this language is expected. To develop machine learning algorithms we will use \texttt{Keras} and \texttt{TensorFlow 2} \cite{chollet:2015keras, tensorflow:2015whitepaper}, which means students must have Python 3.8 installed or a newer version. The university provides the paid version of PyCharm as an IDE for Python, but you can use whatever for writing code.
		
		The required packages for the course can be found in the \texttt{requirement.txt} file in the \texttt{code} folder. From that directory you can install the packages with pip using: \texttt{pip install -r requirements.txt}
		
		The course material is located on GitHub (see the link in the course overview), and students can find the lecture notes in the \texttt{lecture\_notes} folder. The relevant code be found in the subfolders of the \texttt{code} folder.
		
		Students are expected to know calculus, linear algebra, and the basics of neural networks. This course builds on concepts from FY555 (Introduction to Python, machine learning and data handling for the physical sciences), but it is not necessary to have taken that course to follow.
		
	\subsection{Introduction to Git}
		All course material for this course is on GitHub and it is recommended you know how to access this material. GitHub is a developer platform using Git to manage projects. A link to the homepage of Git can be found here: \url{https://git-scm.com/}.
		
		Git is a free and open source distributed version control system. This means that it helps us manage versions of project and collaborate in an asynchronous fashion. Each contributor has a local version of the repository, and can develop locally on that version, and then share changes in a controlled and versioned manner. Git is a key tool for code development and if you ever plan to work in IT, it is something you need to know. It can also be very useful for doing scientific work in groups if code is involved. Also, it allows you to save code and other files in the cloud as a backup. Two of the most popular platforms for using Git is GitHub and GitLab. In this course we use GitHub. Students are encourages to study Git and version control.
		
		There are five main commands you will need to use Git: \texttt{git clone}, \texttt{git pull}, \texttt{git add}, \texttt{git push}, and \texttt{git push}.
		
		To get started you need to download and install Git. Having installed Git you can clone the course repository. This can be done several ways:
		\begin{itemize}
			\item Using a Git plugin in your IDE
			\item With a \texttt{git clone} command in your terminal
			\item By downloading the repository as a zip file (not recommended)
		\end{itemize}
		If you just download the repository as a zip file, you an not leveraging Git. This method works if you simply want to download the course content, but will not teach you how to use Git.
		
		In the following we will consider how to work with Git using the terminal. Assume you want to download the course content from GitHub. Then you would go to the directory where you want to download the repository and execute the following command:\newline \newline
		\texttt{git clone https://github.com/CP3-Origins/advanced-machine-learning.git}
		\newline  \newline
		This will download the repository on your device. Since this course will be updated and improved throughout the course, there might be changes to the course. To update to the latest version you can use the following command anywhere inside the Git project by executing the \texttt{git pull} command.

		This will however not work, if you have locally modified a part of the repository, which you are trying to update. This is to avoid overwriting changes when merging the new file with your local files. In this case you need to resolve any merge issues, such that the file are merged in the desired way without having loosing any work due to file overwriting.
		
		If you plan to participate in exercises and develop your own networks, then it is recommended that you start by creating a fork of the repository. This will create a copy of the repository which is on your own GitHub account. The only downside is that you need to remember update your own fork to get the updates from the main repository on CP3. Other than that you can clone your forked repository and use pull to update your repository.
		
		Suppose you have made your own script with some neural network as part of an exercise, you can now upload that to your fork and save your changes to the cloud. In this case, if you loose your computer, you will still have everything from this course including any code you have created.
		
		To do this you need to push your code from you local machine to your fork on GitHub. To get an overview of file changes in the Git project, you can use the command \texttt{git status}. Then you can add the files you want to save using the command \texttt{git add <path/to/file>}. You can specify a folder or file which you want to add. You can execute \texttt{git add .} if you want all changes to be added. 
		
		Next, you need to commit the code to Git. This is for version control. Together with this it is recommended that you add a short note on what you change does. And example would be \texttt{git commit -m "Adds a network to model rainfall"}, where \texttt{-m} is the message flag and the part in double quotation is the commit message.
		
		Finally, you can use the push command to push the changes using the command \texttt{git push}. This will push your local changes to your fork and you should be able to see the changed if you open your repository on GitHub.
		
		The reason for using a fork is that it creates a personal copy for each student. We cannot have each student to save their work on the main course repository. This would also cause collisions if students used the same file names. You own your fork and you can invite colleagues to your participate if you want to work on the exercise together. 

	\subsection{Neural networks}
		Before going into a more thorough overview of machine learning (ML) methods, let us recall the basics of neural networks and look a implementing neural networks. In the next lecture we will get back to a general discussion of neural networks.
		
		\subsubsection{Theory}
			As the name suggest neural networks are composed of neurons similar to our brain. Any neuron network uses the same basic building block, the single neuron. In some applications such as in robots less than 100 may be sufficient \cite{nagler:2011}, whereas image classification networks may contain thousands of neurons \cite{lorente:2021}. Thus, to understand neural networks we should take our time to understand how they work. Neurons are fundamentally doing linear regression, and later we discuss how nonlinear behavior is obtained as this is key to the power of neural networks. As a simple example, one can consider a single neuron "network" shown in Figure \ref{fig:basicNN}. This "network" takes in three inputs, $x_i$, where $i=1,2,4$, and computes a predicted value $\hat{y}$. The word network is in quotation as one would need multiple interconnected neurons to properly talk of a network. 
			
			\begin{figure}
				\centering
				\begin{neuralnetwork}[height=4]
					\newcommand{\x}[2]{$x_#2$}
					\newcommand{\y}[2]{$\hat{y}_#2$}
					\inputlayer[count=3, bias=false, title=Input\\layer, text=\x]
					\outputlayer[count=1, title=Output\\layer, text=\y] \linklayers
				\end{neuralnetwork}
				\caption{A simple neural network with an input layer, and a single neuron in the output layer.}
				\label{fig:basicNN}
			\end{figure}

			In the case of this single neuron with three inputs shown in Figure \ref{fig:basicNN} the equation becomes,
			\begin{gather}
				z(x_i) = w_1 x_1 + w_2 x_2 + w_3 x_3 + b,
			\end{gather}
			where each input is associated with the weight $w_i$ which modulates the influence of the input, and then there is a bias term, $b$. This can be written in a compact matrix form as,
			\begin{gather}
				z_j(x_i) = 
				\begin{bmatrix}
					w_{11} & w_{21} & w_{31} \\
				\end{bmatrix}
				\begin{bmatrix}
					x_1\\
					x_2 \\
					x_3 \\
				\end{bmatrix}
				+
				\begin{bmatrix}
					b \\
				\end{bmatrix}
				= 
				w_{ij}x_i + b_j \cdot \vec{1},
			\end{gather}
			where an additional $j$ index is added to indicate the number of neurons, but since there is only one neuron it is redundant. The use of this index becomes apparent when considering a network like in Figure \ref{fig:lessbasicNN}, but with two fully connected output nodes, $j=2$. In this case, the equation would be,
			\begin{gather}
				z_j(x_i) = 
				\begin{bmatrix}
					w_{11} & w_{21} & w_{31} \\
					w_{12} & w_{22} & w_{32} \\
				\end{bmatrix}
				\begin{bmatrix}
					x_1\\
					x_2 \\
					x_3 \\
				\end{bmatrix}
				+
				\begin{bmatrix}
					b_1 \\
					b_2 \\
				\end{bmatrix}
				= 
				w_{ij}x_i + b_j \cdot \vec{1}.
			\end{gather}
			
			\begin{figure}
				\centering
				\begin{neuralnetwork}[height=4]
					\newcommand{\x}[2]{$x_#2$}
					\newcommand{\y}[2]{$\hat{y}_#2$}
					\inputlayer[count=3, bias=false, title=Input\\layer, text=\x]
					\outputlayer[count=2, title=Output\\layer, text=\y] \linklayers
				\end{neuralnetwork}
				\caption{A simple neural network with an input layer, and two neurons in the output layer.}
				\label{fig:lessbasicNN}
			\end{figure}
			
			This highlights the benefit of the notation with the weight matrix as $w_{ij}$, the input vector as $x_i$, and the bias vector as $b_j$. A technical note would be that, in practice, it is more efficient to skip the process of adding the bias terms explicitly. One can exploit the definition of matrix multiplication and write it as,
			\begin{gather}
				z_j(x_i) = 
				\begin{bmatrix}
					w_{11} & w_{21} & w_{31} & b_1\\
					w_{12} & w_{22} & w_{32} & b_2\\
				\end{bmatrix}
				\begin{bmatrix}
					x_1\\
					x_2 \\
					x_3 \\
					1 \\
				\end{bmatrix}
				= 
				w_{ij}x_i + b_j \cdot \vec{1}.
			\end{gather}
			
			From this we can then also consider more complex networks with hidden layers as in Figure \ref{fig:NNhidden}. In this case we would first compute the first the values at the hidden layers $z_1(x_i) = a_i $ and then we can use that result to compute that result $z_2(a_i) = \hat{y}$.
			\begin{figure}
				\centering
				\begin{neuralnetwork}[height=4]
					\newcommand{\hidden}[2]{$a_#2$}
					\newcommand{\x}[2]{$x_#2$}
					\newcommand{\y}[2]{$\hat{y}_#2$}
					\inputlayer[count=3, bias=false, title=Input\\layer, text=\x]
					\hiddenlayer[count=3, bias=false, title=Hidden\\layer, text=\hidden] \linklayers
					\outputlayer[count=1, title=Output\\layer, text=\y] \linklayers
				\end{neuralnetwork}
				\caption{A simple neural network with an input layer, a hidden layer, and a single neuron in the output layer.}
				\label{fig:NNhidden}
			\end{figure}
			
			A linear equation is not sufficient to produce complex nonlinear decision boundaries, therefore it is key to modify the result $z_j(x)$ such nonlinearity is achieved. This is very simply to prove, consider our simple network with one hidden layers. The math becomes,
			\begin{gather}
				\hat{y} = z_2(z_1(x_i)) = w_{i2}\paa{w_{i1}x_i + b_1 \cdot \vec{1}} + b_2 \cdot \vec{1}.
			\end{gather}
			The problem here is that while we have different weights and biases, due to the lack of nonlinearity, this hidden actually achieve nothing because we can redefine this equation to be just like a network without a hidden layer,
			\begin{gather}
				\hat{y} = z_2(z_1(x_i)) = \underbrace{w_{i2}w_{i1}x_i}_{w_{i1}x_i} + \underbrace{w_{i2}b_1 \cdot \vec{1} + b_2b_1 \cdot \vec{1}}_{b_1 \cdot \vec{1}}.
			\end{gather}
			This is perfectly valid because the latter terms is nothing but a sum of trainable variable, which is nothing but a number, and then first part contains the input times two different weight, but again we can just redefine that as a single weight. Therefore, without any nonlinearity we can add an arbitrary number of hidden layers and achieve the equivalent of having no hidden layers.
			
			This is fixed using nonlinear activation functions, $a(z)$, such as the sigmoid function, $\sigma(x)$, and the rectifier linear (ReLu) function, $\rm{ReLu}(x)$. There are many more options, and it is up to the designer of the network to determine the best activation function. The activation function, $f$, can be applied on a per-neuron basis, but it is commonly applied across the layers,
			\begin{gather}
				a(z_j(x_i)) = f(w_{ij}x_i + b_j \cdot \vec{1}).
			\end{gather}
			Each layer of the neural network is generally computed in this way. One can then stack the layers, $a(z_j)^n$, where $n$ is the number of layers. Deep neural networks then have three layers or more, $n \ge 3$. The output of the neural network is then computed by iteratively computing the layers from input to output, with the previous layer being the input to the next layer.
			
		\subsubsection{Tensorflow and keras}
			For this course we will be using \texttt{Keras} and \texttt{TensorFlow 2} \cite{chollet:2015keras, tensorflow:2015whitepaper} to develop neural network. There are also other packages such as \texttt{PyTorch} \cite{paszke2017automatic} which you are free to use, but it will not be used in the example code of this course. At the end of the day it is all linear algebra, but there are some differences in the packages. In general \texttt{Keras} provides a more beginner-friendly interface, but we will introduce advanced features which can be done in both \texttt{TensorFlow} and \texttt{PyTorch}. Some of the exercises are will be suitable to do in \texttt{PyTorch} and you are welcome to do so.
			
			This weeks exercise is about introducing the basics of coding with \texttt{Keras} and \texttt{TensorFlow 2}. This example has also been used in FY555, but in this course we will consider a different exercise.
			
			For this first part we will use \texttt{Keras} and some \texttt{sklearn} functions to started. First we import dependencies:			
			\begin{lstlisting}[language=Python]
import matplotlib.pyplot as plt
from sklearn.datasets import load_iris
from sklearn.model_selection import train_test_split
from sklearn.preprocessing import StandardScaler, LabelEncoder
from keras.models import Sequential
from keras.layers import Dense
from keras.utils import to_categorical
			\end{lstlisting}
			Then we download the iris dataset from \texttt{sklearn} as our simple training set, and convert the text labels to numbers using encoding.
			\begin{lstlisting}[language=Python]
# Data (as pandas dataframes)
X = load_iris().data
y = load_iris().target

# Convert string labels to numeric values using Label Encoding
label_encoder = LabelEncoder()
y = label_encoder.fit_transform(y)

# Convert labels to one-hot encoding for categorical crossentropy
y_one_hot = to_categorical(y)
			\end{lstlisting}
			Next step is to split the dataset into training and test. The input data is then normalized before testing.
			\begin{lstlisting}[language=Python]
# Split the data into training and testing sets
X_train, X_test, y_train, y_test = train_test_split(X, y_one_hot, test_size=0.2, random_state=42)

# Standardize the features (optional but often recommended)
scaler = StandardScaler()
X_train = scaler.fit_transform(X_train)
X_test = scaler.transform(X_test)
			\end{lstlisting}
			In the following section we create the network using the \texttt{Sequential} method. The network consists of a hidden layer with 8 nodes and a final layer of three nodes for each possible classification category. For the hidden layer ReLu functions are used, and for the final layer the softmax activation function is used for classification.
			\begin{lstlisting}[language=Python]
# Create a sequential model
model = Sequential()

# Add the input layer and the first hidden layer
model.add(Dense(units=8, input_dim=4, activation='relu'))

# Add the output layer with softmax activation for classification
model.add(Dense(units=3, activation='softmax'))
			\end{lstlisting}
			After defining the model we compile it with an optimizer, in this case the \texttt{adam} optimizer, and then we define a loss function and metrics. Then we train the model with fit and split the training dataset into the training and validation set. Then train with batch sizes of 8 across 200 epochs.
			\begin{lstlisting}[language=Python]
# Compile the model with categorical crossentropy loss for multi-class classification
model.compile(optimizer='adam', loss='categorical_crossentropy', metrics=['accuracy'])

# Train the model
history = model.fit(X_train, y_train, epochs=200, batch_size=8, validation_split=0.2)
			\end{lstlisting}
			Having trained the model, we can evaluate the performance using the test dataset and then print the result.
			\begin{lstlisting}[language=Python]
# Evaluate the model on the test set
loss, accuracy = model.evaluate(X_test, y_test)
print(f"Test Loss: {loss}, Test Accuracy: {accuracy}")

# Plot training and validation loss over epochs
plt.figure(figsize=(12, 6))

# Plot training & validation loss values
plt.subplot(1, 2, 1)
plt.plot(history.history['loss'])
plt.plot(history.history['val_loss'])
plt.title('Model Loss')
plt.xlabel('Epoch')
plt.ylabel('Loss')
plt.legend(['Train', 'Validation'], loc='upper right')

# Plot training & validation accuracy values
plt.subplot(1, 2, 2)
plt.plot(history.history['accuracy'])
plt.plot(history.history['val_accuracy'])
plt.title('Model Accuracy')
plt.xlabel('Epoch')
plt.ylabel('Accuracy')
plt.legend(['Train', 'Validation'], loc='lower right')

plt.tight_layout()
plt.show()
			\end{lstlisting}
			This example will output how the loss and accuracy improves across the training. Note, that this problem is linear and a neural network is a bit overkill. This simple example is to show all the parts needed to have a functional setup with plotting of the training process. We will build on this in further lectures.

	\subsection{Exercises}
		The exercises this week aim to introduce the basic ways of creating neural networks. We will take a look at three methods, with the latter one being more advanced and what you will need to master to have maximal customization control of your network. Relevant code for this week is found in the folder \texttt{week 1}. The network you need to recreate is the network we discussed in the previous section with one hidden layer with 8 nodes, 3 nodes in the final layer, and 4 input parameters. A working version of this network can be found in Python script \texttt{main.py}. This network you will have to recreate in different ways in this weeks exercises. If you can replicate the result from this example code using the method specified in the exercise you can consider the exercise complete. If you need help, consult the official documentation. \newline
		
		\textbf{Exercise 1: Sequential class}\newline
		In this exercise you need to use the sequential class as in this weeks example code. However, instead of using the \texttt{model.add} method you should create a list of layers and parse them directly to the sequential class. You should thus have a model definition like \texttt{model = Sequential([your\_list])}. For this exercise the script \texttt{sequential.py} has been prepared and you need to implement code where the \texttt{TODO} comment is located.\newline
		
		\textbf{Exercise 2: Functional interface}\newline
		In this exercise you need to use the functional interface. Here you need to define your input as a layer and then the layers with neurons. Then you can define the model as \texttt{model = Model(inputs=input\_layer, outputs=x)}. For this exercise the script \texttt{functional.py} has been prepared and you need to implement code where the \texttt{TODO} comment is located.\newline
		
		\textbf{Exercise 3: Subclassing the model class}\newline
		In this exercise you need to use Subclassing, where you create your network as a classing inheriting from the model class. Here you need to define your class as a subclass for the model class, e.g. \texttt{class Network(Model)}. Then you need to declare your layers in the \texttt{\_\_init\_\_} function and the then build the network by defining a function called \texttt{call} which takes \texttt{self}, and the input tensor as variables. Remember to return the result. Then you can define the model as \texttt{model = Network()}. For this exercise the script \texttt{subclassing.py} has been prepared and you need to implement code where the \texttt{TODO} comment is located.\newline
		
\section{Machine learning and methods of training}
	In this lecture we look at machine learning in a broader perspective to get a feeling for the different methods which one can use. Then we focus in on neural networks and deep learning, before looking at methods of training neural networks.
	
	\subsection{Machine learning}
		Before we go into advanced concepts of neural networks, we better take a step back to understand the bigger picture. What is the difference between machine learning (ML), artificial intelligence (AI), neural networks (NNs), and deep learning (DL)?
		
		As it turns out, neural networks is an outlier in this list. Neural networks is a particular method of machine learning, just like linear regression and decision trees. 
		\begin{figure}
			\centering
			\includegraphics[width=0.7\linewidth]{figs/ai}
			\caption{An overview of the connection between AI, ML, DL. Note that neural networks are not mentioned.}
			\label{fig:ai}
		\end{figure}
		
		As shown in Figure \ref{fig:ai}, AI is the overarching name for algorithms that aim to imitate some kind of intelligence. A very simple AI that does not use any kind of fancy ML methods would be something like the following code:
		\begin{lstlisting}[language=Python]
if temperature > 20: 
	return 0
else:
	return 1
		\end{lstlisting}
		This example code is very simple, but it shows a simple AI algorithm which for example could be used to control temperature based on some temperature reading. One can make a more complex logic, but this is how one can make basic AI algorithms, and such algorithms have often been used in the past for various of applications to control sensors or bots in computer games.
		
		Machine learning is a more advanced subset of AI methods where the algorithm can learn from data. Such algorithms do not require developers to hard code logic into the algorithms, as the algorithm will create its own logic after been trained on data. For complex methods such as neural networks it leads to the black box issue, that the developer does not understanding how the algorithm makes its decisions, but can merely try to assert if the decision is correct or not.
		
		To give you an idea of ML methods here is a list of some well-known methods:
		\begin{itemize}
			\item Artificial neural networks
			\item Decision trees
			\item Genetic algorithms
			\item Support-vector machines
			\item Regression analysis
			\item Bayesian networks
			\item Gaussian processes			
		\end{itemize}
		These different methods have different areas of applications, and many of them a valid in the same application areas. The training requirements may also differ for each method, and many of these methods can often be designed to trained in different ways.
		
		A short word on deep learning. Deep learning is when we talk of more complex ML, meaning not just to a simple densely connected neural network with a few layers, but rather models with thousands if not millions of parameters. Such models can often combine different ML methods and different training methods for the different components of the DL model. An example is a convolutional network that uses convolution and pooling layers to condense a lot of information into a densely connected network.
	
	\subsection{Training of ML methods}
		Since the point of ML methods is to learn from data, and not work by parameters set by humans, a key part to this field is the way you train your ML method. In this section we will look at three main ways of training ML methods.
		
		\subsubsection{Supervised learning}
			On of the most popular ways of training neural networks is supervised training. It is very simple to do, and with more data you can generally make an increasingly good ML model with supervised learning and a ML method like a neural network. In this section we consider neural networks. Some of the pros of this method is: 
			\begin{itemize}
				\item For classification we can decide the number of categories
				\item You can mimic anything given enough labeled data
				\item A trained model is often very resource effective
				\item We can have a good idea of the accuracy of the model
			\end{itemize}
			One the other hand it can also have the following downside:
			\begin{itemize}
				\item Requires a lot of labeled data
				\item Poor performance outside of the scope of the training
				\item Cannot detect new categories
				\item Training is time consuming
			\end{itemize}			
			Consider an arbitrary function $f(x)$ of input $x$. The result of this will some value $y$. We are thus considering a function that maps and input $x$ to $y$,
			\begin{gather}
				f: x \rightarrow y.
			\end{gather}
			Given enough data we can train a neural network to become any function $f(x)$ and thus have the neural network imitate whatever we want. The main requirement is that our neural network have enough degrees of freedom and that we have an appropriate amount of training data. 
			
			In the case of regression problems we can often use a neural network. In some cases we might be model something using a very complicated or compute intensive mathematical model, fx fluid dynamics. In this case it is possible to make an approximate model with a neural network. It can often be orders of magnitude faster to compute a bit of matrix multiplication than solving differential equations, and with enough data one can get very close with the approximation one can do with a neural network, keeping mind that the alternative model often also uses approximations to increase compute performance \cite{fukami:2019, wang:2019, vandal:2017, ling:2016}.
			
			Another case is classification and image recognition, and this is a huge field where supervised learning is showing potential. This has many practical applications. Consider making a self-driving car, one thing to do is to recognize the surroundings. There could be trees, people, other cars and many other obstacles. 
			
			Consider making to mathematical model of how a car or humans look. It's impossible to do with a simple description, because cars can be of different types such as SUV's, micro cars, hatchbacks, estates, vans etc. and the color could also be different together with other factors. Likewise, humans comes in different sizes ans shapes and with different skin color, clothes. There is no simple description for such objects.
			
			Instead of trying to model this with some complex function, we can just use a neural network and train it with supervised learning if we have enough labeled training data. We don't have to know the underlying model, it's hidden inside the neural networks in the different weights and biases. As long as we have all types of cars, humans, etc. represented in the training data, it should be possible to get a pretty good classification from a neural network. This we can then use to figure out the environment around our self-driving car and take act. Should we brake because there is a pedestrian in front of us, or is that a car in front? Together with radar sensors to measure and attach velocities to the objects around us we can react to the environment. One can also use AI to recognize speed signs etc. to understand the speed limit or other signs.
			
			\begin{figure}
				\centering
				\begin{tikzpicture}
					\begin{axis}[%
						axis lines = left,
						xlabel = $x$,
						ylabel = $y$,
						clip mode=individual % so things drawn by \draw and similar are not cut off
						]
						\addplot [blue, only marks, mark=*, mark size=5] table [%
						x = x, 
						y = y, 
						col sep = comma]{
							x, y
							%cluster 1
							2, 3
							3, 5
							4, 5
							3, 8
							5, 9
							3, 2
							5, 6
							6, 6
							7, 9
							10, 4
							11, 5
							9, 4
						};
						
						\addplot+[red, only marks, mark=*, mark size=5] table [%
						x = x, 
						y = y, 
						col sep = comma]{
							x, y
							20, 10
							21, 12
							24, 12
							25, 13
							27, 14
							22, 13
							23, 15
							25, 10
							15, 14
						};
						
						% to be able to use axis coordinates with \draw directly you need
						% \pgfplotsset{compat=1.11} or a higher version
						% if that is not present, use (axis cs:4,14) instead of (4,14),
						% to specify that the values should be interpreted as axis coordinates
						\draw [dashed] (4,14) -- (25,2);					
					\end{axis}
				\end{tikzpicture}
				\caption{Boundary division from supervised learning.}
				\label{fig:supervised}
			\end{figure}
			Without ML methods it would be near impossible to make a self-driving car, but it seems like it can more or less be done with machine learning. But we also see problems to work on, such as handling things that the networks hasn't been trained to handle.
			
			This is one of the limitations, there is no guarantee that the network will perform well outside the scope of the training. The logic could be very divergent outside of the training area causing very dangerous driving in the case of our self-driving car, and it cannot detect new categories and somehow evolve without retraining.	
		
			In Figure \ref{fig:supervised} it is illustrated how supervised learning aims a optimizing the boundary between the categories we have defined. Given we have activation function this boundary can be non-linear and generally it is non-linear, but is shown linear for simplicity.
				
		\subsubsection{Unsupervised learning}
			An alternative to supervised learning is unsupervised learning. There are many ML methods which can be used in this context including neural networks such as autoencoders. The main goal of unsupervised learning is to train an algorithms without having any labeled training data, in general you just have some result, $y$, but not the input $x$ that led to the result.
			
			Some of the cons of this method can be listed as:
			\begin{itemize}
				\item It can construct categories by itself
				\item New features and patterns can be discovered
				\item No need for labeled data
			\end{itemize}
			One the other hand it can also have the following downside:
			\begin{itemize}
				\item Interpretation is require to understand the detected patterns
				\item Accuracy is often worse than supervised learning
				\item No guarantee a useful result will be obtained
			\end{itemize}

			\begin{figure}
				\centering
				\begin{tikzpicture}
					\begin{axis}[
						minor tick num=3,
						%  axis y line=left,
						%  axis x line=bottom,
						xlabel=$x$,ylabel=$\sin x$,
						%  yticklabels=\empty,
						ymax=2,
						ymin=-2,
						background/.style={fill=gray},
						]
						\addplot[smooth,blue,mark=none, domain=0:10,samples=40]
						{cos(deg(2*x))};
						\addplot[dashed,red,mark=none, domain=5.5:7.1,samples=40]
						{2*cos(deg(2*x))};
					\end{axis}
				\end{tikzpicture}
				\caption{Anomaly detection using unsupervised learning. The expected result is the blue line, but then there might be a peak like the dashed red line.}
				\label{fig:anomaly}
			\end{figure}
					
			Since unsupervised learning doesn't require any knowledge input from the creator, it will try to find logic in any data given. It also means it is sensitive to anomalies and can thus be used to anomaly detection from data.
			
			Consider some sinus curve for simplicity as seen in Figure \ref{fig:anomaly}. You might have some analogue signal of this shape, and you want to monitor if there is something unusual going on with signal. You can of course manually make checks if the signal has a certain amplitude, period etc. But in less simple cases like monitoring of internet packages or or complex flows, it might not be trivial with hand made definitions what is normal and what is an anomaly.
			
			The point is we can use machine learning methods to learn and by itself define some expected patterns of a given input. Then if suddenly an input yields something unexpected the algorithm can flag it, and thus, detect the anomaly without any human definition of what an anomaly is. In this way we don't have any human bias in determining the parameters of the anomaly and we may detect patterns that a human could have missed. 
			
			Generally, we can again consider the case of two different populations like with supervised learning. In the former case we look at optimizing a boundary between the predefined categories. In the case of unsupervised learning the grouping is done based on finding common ground in the data and thus finding a cluster of related points. This clustering is illustrated in Figure \ref{fig:unsupervised}, and one can also see some of the dots falling outside the clusters either suggesting some noise or some anomaly depending on the physical background of the data. If the errors are under control than we could argue that the red point outside the circle is an anomaly.
			
			\begin{figure}
				\centering
				\begin{tikzpicture}
					\begin{axis}[%
						axis lines = left,
						xlabel = $x$,
						ylabel = $y$,
						clip mode=individual % so things drawn by \draw and similar are not cut off
						]
						\addplot [blue, only marks, mark=*, mark size=5] table [%
						x = x, 
						y = y, 
						col sep = comma]{
							x, y
							%cluster 1
							2, 3
							3, 5
							4, 5
							3, 8
							5, 9
							3, 2
							5, 6
							6, 6
							7, 9
							10, 4
							11, 5
							9, 4
						};
						
						\addplot+[red, only marks, mark=*, mark size=5] table [%
						x = x, 
						y = y, 
						col sep = comma]{
							x, y
							20, 10
							21, 12
							24, 12
							25, 13
							27, 14
							22, 13
							23, 15
							25, 10
							15, 14
						};
						
						% save a coordinate for use later
						\coordinate (c2) at (23,12);
						
						% the blue circle is drawn inside the axis environment, and in axis coordinates
						% hence it becomes an ellipse
						\draw [blue, dashed] (6,6) circle[radius=5]; 
						
					\end{axis}
					% the red circle is drawn outside the axis, so actually looks like a circle,
					% but the radius has no relation to the axis coordinates
					\draw [red, dashed] (c2) circle[radius=2cm];
				\end{tikzpicture}
				\caption{Clustering of self learned groups from unsupervised learning.}
				\label{fig:unsupervised}
			\end{figure}
			
			Another example of unsupervised learning that is a be different is that of autoencoder neural networks (and the similar variational autoencoder). This type of network have the same input as output, or some slightly perturbed input from the output.
			
			The idea is that you take in an input and then reduce it to some latent vector representation of the input, that is the encoder part of the network. Then the decoder part of the network then recovers the input from the latent vector. You might ask why you would want to do this, and there are a few reasons.
			
			The latent vector in the middle is a compact representation of some input. You could think of simple example like that of irises. Let's say you have a stack of pictures of 4 different types of iris flowers. Now, the input is the picture and so is the output. But then in principle, you should be able to compress the input to a vector of 4 degrees of freedom for each type of irises. In the case you didn't know that, you might be able to obtain a minimal categorization of whatever input you consider, because the latent vector contains the minimal information to recover the input again.
			
			Another example is that of denoising of images. You can easily take a lot of images, add a bit of Gaussian noise to the input and demand a clear and sharp output. In this case you are can train the autoencoder to denoise images or perform any other systematic transformation of an image.
		
		\subsubsection{Reinforcement learning}
			Somewhere between supervised and unsupervised learning we have the method called reinforcement learning. In this case we don not require a labeled dataset, but we are to define a reward function to maximize. It is incredible powerful and one of the methods also used in ChatGPT together with supervised learning.
			
			This result in some of the following cons of this method:
			\begin{itemize}
				\item It can self correct errors
				\item it can learn from experience
				\item Very similar to human learning
				\item No need for labeled data
			\end{itemize}
			One the other hand it can also have the following downside:
			\begin{itemize}
				\item Not suitable for simple problems
				\item It is data and compute hungry
				\item It assumes a Markovian world
				\item Depends on the validity of the reward function
			\end{itemize}
		
			The fact reinforcement learning does not require labeled data is a huge advantage, but it does require a lot of data to train. This type of learning is similar to how humans think. We often have some values that we consider when we make decisions. How can I maximize profit? How can I make my life as good as possible. This is essentially the same as what we want to achieve with the reward function. Then the training measures different inputs and how such actions help achieving the goal encoded in the reward function.
			
			As mentioned, reinforcement learning may face issue from the fact that the world is not Markovian which means that the next state depends only on the previous state, which is an oversimplification and in general we humans sometimes do things that make no sense, which again does not fit into this picture. Nevertheless, this approach is good enough, and using neural networks with memory one can future counter such issues. 
			
			A key concept of reinforcement learning is exploration. Let's say you want to make a robot that can walk, but you have no idea how to program that. You could define a reward function stating, the further you move, the bigger the reward. In principle this is simple and a valid loss function. There is however a big issue.
			
			If you just put your stationary robot on the table and tell it to learn it will do nothing. Why? Well, it's not actually doing any changes that could lead to a better state. This leads to the importance of exploration.
			To make the robot learn we need to allow it to do something random to try out what could possible work, this is why we will modify the action to do with some perturbation $\epsilon$ called exploration. The probability of the robot doing what it believes is the best is thus $1-\epsilon$. This, means that if the exploration is high, there is a high probability the robot will not do the action it considers to be best. This is good in the beginning, because the robot will not know what is good, but as it learns we will likely want to decrease the value of $\epsilon$ as we converge to the optimal action. With exploration on can make a robot with a neural network walk just from defining a reward function based on how far the robot walks.
			
			Another key to remember, if you are training a neural network with reinforcement learning, then ensure that the reward has a dependency on weights, otherwise the weights will not be updated using gradient descent as the derivative of the update is zero.
			
	\subsection{Training of neural networks}
		Focusing on how to train neural networks it is necessary to update the trainable parameters of the network which is the weights and biases. Whatever training method you are using, there will be some input to the network. In the case of supervised learning there will also be some output from the training data. The result is that we compute some loss or reward function to determine how we should update the network. Note that if you have some reward function, you need to make sure you end up with a loss function that can be minimized if you want to train the network as described in this section.
		
		One of the key concepts to train neural networks is the loss function. For this we consider the case of supervised learning. This is a function that measures the deviation of our prediction and thus how wrong the result from the network is. There are different functions one can use, but one simple example would be the mean squared error (MSE) loss,
		\begin{gather}
			\mathcal{L}(\hat{Y}, Y) = \frac{1}{n} \sum_{n=1}^{n} \paa{\hat{y}_i - y_i}^2 =  \frac{1}{n}\paa{\hat{Y} - Y}^2,
		\end{gather}
		where we are summing over the $n$ values of the output vector $Y$ and predicted output vector $\hat{Y}$. Likewise one could also consider the mean absolute error (MAE),
		\begin{gather}
			\mathcal{L}(\hat{Y}, Y) = \frac{1}{n}\abs{\hat{Y} - Y},
		\end{gather}
		where we simple consider the difference of the prediction and true value. You can define whatever value you want, but some loss functions might be more suitable than others. The MSE might not always be the best if the error is less than one, because the squared value yields a smaller loss than just taking the absolute loss, but often the MSE loss is quite effective.
		
		This loss function is then a measure in our cost function, where we consider the total cost of our model. Note, that you can have more than just a single loss function as part of the cost function. We can write the cost function as a function of the underlying trainable weights $\Theta = \{w_{ij}, b_j\}$ which is composed of weights and biases,
		\begin{gather}
			J(\Theta) = \frac{1}{m} \sum_{m=1}^{k} \mathcal{L}(\hat{Y}, Y)_k,
		\end{gather}
		where we sum over the batch size $m$. The cost function is then used for gradient descent updating of the weights and biases in the following ways,
		\begin{gather}
			w_{ij}' = w_{ij} - \alpha \frac{\pd J}{\pd w_{ij}},\\
			b_{j}' = b_j - \alpha \frac{\pd J}{\pd b_{j}},
		\end{gather}
		where $\alpha$ is the learning rate. Now I should note that this math describes the simple gradient descent optimization, but we often use a different optimizer such as the one called Adam \cite{kingma:2017}. This algorithm introduces concepts such as momentum, which helps to effectively adjust the learning rate along the way.

	\subsection{Exercises}
		In this week we will consider exercises related to training methods. We look at supervised and reinforcement learning and compare them. Here we make use of Python classes, though you do not have to worry about this. It is just to show more advanced use of Python, and for now you don't have to program more advanced customization, but your are encouraged to study the whole code. If you can understand all that is going on, you are doing well. The code for this week and be found in the folder \texttt{week 2}.\newline
		
		\textbf{Exercise 1: Reinforcement learning}\newline
			In this task we consider solving a few ordinary differential equations (ODEs). We consider the range $x \in [0,2]$. In particular we want to solve the following equations,
			\begin{gather}
				\frac{\pd u}{\pd x} = 2x, \quad u(0) = 1,\\
				\frac{\pd u}{\pd x} = x^2, \quad u(0) = 1,\\
				\frac{\pd u}{\pd x} = x^2 - 2x, \quad u(0) = 1.\\
			\end{gather} 
			The idea is that we can make a neural network $NN(x)$ be an approximate function of $u(x)$, thus $NN(x) \approx u(x)$. To train it we consider the definition of the derivative,
			\begin{gather}
				\frac{\pd NN}{\pd x}= \lim_{h\rightarrow 0} \frac{f(x+h) - f(x)}{h},
			\end{gather} 
			and note $h$ as a infinitesimal exploration parameter. This allows us to write out loss function as an MSE loss, 
			\begin{gather}
				\mathcal{L} = \paa{\frac{\pd NN(x)}{\pd x} -\frac{\pd u}{\pd x}}^2,
			\end{gather}
			now to incorporate the initial condition we can make the substitution, 
			\begin{gather}
				g(t) = u(0) + x NN(x),
			\end{gather} 
			and obtain the loss function,
			\begin{gather}
				\mathcal{L} =\paa{\frac{\pd g(x)}{\pd x} -\frac{\pd u}{\pd x}}^2.
			\end{gather}
			This loss function is already implemented together with a custom training method in the file \texttt{reinforcement.py}.
			What you need to do is to understand what is going on, tune the hyperparamters, and make the network where \texttt{TODO} comment is located such that it can we can obtain an accurate approximation. You can use the plot that is created to see if you are close.\newline
		
		\textbf{Exercise 2: Supervised learning}\newline
			Unfortunately, it is not possible to solves ODEs with supervised learning as it requires labeled $y$ values to train. Fortunately, we know the analytical solutions.  We consider the range $x \in [0,2]$ again. In this exercise you must create a network and compile is as \texttt{self.model} where \texttt{TODO} comment is located in the file \texttt{supervised.py}. You must train it and use the data created for you as the parameters $x$ and $y$. You are free to build and train the model as you like. The goal is to make a supervised model that perfectly imitates the analytical solution.
			\newline
			
		\textbf{Exercise 3: Out of your comfort zone?}\newline
			As we discussed before, supervised learning in particular is not going to guarantee a good result outside of the training zone. In this task you need to compare the result of using your models from the reinforcement and supervised learning exercise and compare the result when trying to apply them to inputs of the range $x \in [0,5]$. Note that they should still be trained on the range $x \in [0,2]$. The code for this task can be found in \texttt{compare.py} and you need to create your data and plot it \texttt{TODO} comment is located.
			
			Compare the two machine learned model and the analytical result for all three solutions in a nice plot. What does this tell you about the performance of supervised and reinforcement learning models outside the training scope?
		
\section{Neural network architectures}\label{sec:lecture3}
	In this lecture we consider architectures of neural network and look at alternatives to densely connected neurons. In particular we take a look at convolutional neural networks (CNNs) and recurrent neural networks (RNNs) together with other methods of building networks with memory. The goal is to get a feel for the different types of layers and components we can use to build a network. 
	
	To this end, it is important that we carefully consider the problem at hand as this will help us understand how to go about designing a network. If we have a large input or output then maybe we should consider element of CNNs. Alternatively, if we consider applications requiring memory, often in cases with time evolution, then RNN or other methods that can store memory will be good to consider. Often it can be meaningful to combine the different types of layers and methods.
	
	\subsection{Artificial neural networks}
		Traditional artificial neural networks (ANNs) are very effective in many applications providing a function that can approximate effectively and cheaply \cite{lin:2017}. 
		
		When an ANN has three or more layers one can consider it as a deep neural network (DNN). This, arguably, is not enough to qualify the network as a DL method, but it is a step in that direction. DNNs can contain tens of layers if not even more in order to have enough parameters to model complex behavior. Having a very wide network with many neurons in each layer is one way to add degrees of freedom, but one may reduce the number of neurons per layer and have a longer network with many smaller layers. Since the layer parameters scale as the product of the number of nodes in the previous layer times the number in the current layer, one can quickly end up with a lot of parameters to train for wide networks. This fact is illustrated in Figure \ref{fig:DNN}, where the wide network in Figure \ref{fig:DNNwide} as 33 parameters compared to the narrow network in Figure \ref{fig:DNNnarrow} with 25 parameters. This includes the weight parameters, each weight parameter corresponds to a connection, plus one bias parameter from each node. Both networks have the same number of neurons, but not the same number of parameters. This is something the architect of the network has to consider in terms of what effects the width has on the output and the trainability of the network.
		
		\begin{figure}
			\centering
			\begin{subfigure}[b]{0.49\textwidth}
				\centering
				\begin{neuralnetwork}[layerspacing=10mm, height=4]
					\newcommand{\x}[2]{$x_#2$}
					\newcommand{\y}[2]{$\hat{y}_#2$}
					\newcommand{\hfirst}[2]{\small $a^{[1]}_#2$}
					\newcommand{\hsecond}[2]{\small $a^{[2]}_#2$}
					\inputlayer[count=1, bias=false, title=Input layer, text=\x]
					\hiddenlayer[count=4, bias=false, title=Layer 1, text=\hfirst] \linklayers
					\hiddenlayer[count=4, bias=false, title=Layer 2, text=\hsecond] \linklayers
					\outputlayer[count=1, title=Output layer, text=\y] \linklayers
				\end{neuralnetwork}
				\caption{A simple neural network with wide layers. This network 33 trainable parameters.}
				\label{fig:DNNwide}
			\end{subfigure}
			\hfill
			\begin{subfigure}[b]{0.49\textwidth}
				\centering
				\begin{neuralnetwork}[layerspacing=10mm, height=2]
					\newcommand{\x}[2]{$x_#2$}
					\newcommand{\y}[2]{$\hat{y}_#2$}
					\newcommand{\hfirst}[2]{\small $a^{[1]}_#2$}
					\newcommand{\hsecond}[2]{\small $a^{[2]}_#2$}
					\newcommand{\hthird}[2]{\small $a^{[3]}_#2$}
					\newcommand{\hfourth}[2]{\small $a^{[4]}_#2$}
					\inputlayer[count=1, bias=false, title=Input layer, text=\x]
					\hiddenlayer[count=2, bias=false, title=Layer 1, text=\hfirst] \linklayers
					\hiddenlayer[count=2, bias=false, title=Layer 2, text=\hsecond] \linklayers
					\hiddenlayer[count=2, bias=false, title=Layer 3, text=\hthird] \linklayers
					\hiddenlayer[count=2, bias=false, title=Layer 4, text=\hfourth] \linklayers
					\outputlayer[count=1, title=Output layer, text=\y] \linklayers
				\end{neuralnetwork}
				\caption{A simple neural network with many narrow layers. This network 25 trainable parameters.}
				\label{fig:DNNnarrow}
			\end{subfigure}
			\caption{Two architectures with the same number of nodes, but with a different number of parameters.}
			\label{fig:DNN}
		\end{figure}
		
		This example demonstrates the concept that wide layers cost more to train, as you have more parameters to train. For many applications, a DNN with a sufficient number of layers can prove very successful, but the hidden layers are nothing but a black box, a common problem with neural networks. The larger the network, the harder it becomes to interpret the logic of the network, and in practice, it is generally not possible to understand the logic the network has learned.
		
		Regarding the issue of training parameters, one can consider a problem in which the input is very large. In a DNN each input is densely connected to the first layer, thus for each neuron in the first layer, one can multiply the number of weights required to process the input. Furthermore, large inputs usually require a rather large first layer to retain the information from the input layer, thus the larger the input the larger the first layer should be leading to the problem of too many parameters. This is one of the limiting factors of DNNs, together with the fact that the neurons have no memory to store information about the previous value it processed.
		
		\subsection{Convolutional Neural Networks}
		A solution to the large input problem is the convolutional neural network (CNN). The problem is that dense connections can result in an exploding number of parameters, which become computationally unfeasible to train. Luckily, it isn't necessary to always use dense layers and a way to reduce the number of parameters is using CNN which uses convolution and pooling to reduce the problem, while retaining key information, before handing it over to a DNN network to produce the desired output. In the case of networks such as autoencoders, one might upscale the dimension of the output to match that of the input by doing upsampling instead of pooling.
		
		In this section, the focus is the convolution and pooling used in CNNs. These networks are often used in image processing, thus, it makes sense to consider the case of a 2D input. Instead of connecting every input to every single neuron in the first layer, one can use a kernel. This means each neuron in the layer is only connected to a subspace of the full input. The size of the kernel is thus the size of the weights of each neuron, and one can still have a bias term. By having for example a three by three kernel, as shown in Figure \ref{fig:convolution}, one would end up with only 9 weights per node plus the bias instead of e.g. 49 weights if the full input is a 7x7 two dimensional input. The reduction here is limited because the input as shown in Figure \ref{fig:convolution} is not very large due to practical reasons, but consider a modern 43.2 MP picture with dimensions of 6000x7200. In this case, the convolution neuron would still have 9 weights and not 43.2 million weights as the case would be for a fully connected neuron.  
		
		\begin{figure}
			\centering
			\begin{tikzpicture}
				% ------- style -------
				\tikzset{%
					parenthesized/.style={%
						left delimiter  = (,
						right delimiter = ),
					},
					node distance = 10mu,
				}
				
				% ------- equation -------
				\matrix[matrix of math nodes, parenthesized] (I) {
					0 & 1 & 1 & 1 & 0 & 0 & 0 \\
					0 & 0 & 1 & 1 & 1 & 0 & 0 \\
					0 & 0 & 0 & 1 & 1 & 1 & 0 \\
					0 & 0 & 0 & 1 & 1 & 0 & 0 \\
					0 & 0 & 1 & 1 & 0 & 0 & 0 \\
					0 & 1 & 1 & 0 & 0 & 0 & 0 \\
					1 & 1 & 0 & 0 & 0 & 0 & 0 \\
				};
				
				\node (*) [right = of I] {${}\cross{}$};
				
				\newcommand\Kmatrix{}
				\foreach \row in {1, ..., 3} {
					\gdef \sep {}
					\foreach \col in {1, ..., 3} {%
						\xdef \Kmatrix {\unexpanded\expandafter{\Kmatrix}\unexpanded\expandafter{\sep}\noexpand \K{\row}{\col}}
						\gdef \sep { \& }
					}
					\xdef \Kmatrix {\unexpanded\expandafter{\Kmatrix}\noexpand\\}
				}
				\matrix[matrix of math nodes, parenthesized, ampersand replacement=\&] (K) [right = of *] {
					\Kmatrix
				};
				
				\node (=) [right = of K] {${}={}$};
				
				\matrix[matrix of math nodes, parenthesized] (I*K) [right = of {=}] {
					1 & 4 & 3 & 4 & 1 \\
					1 & 2 & 4 & 3 & 3 \\
					1 & 2 & 3 & 4 & 1 \\
					1 & 3 & 3 & 1 & 1 \\
					3 & 3 & 1 & 1 & 0 \\
				};
				
				% ------- highlighting -------
				\newcommand\rowResult{1}
				\newcommand\colResult{4}
				
				\begin{scope}[on background layer]
					\newcommand{\padding}{2pt}
					\coordinate (Is-nw) at ([xshift=-\padding, yshift=+\padding] I-\rowResult-\colResult.north west);
					\coordinate (Is-se) at ([xshift=+\padding, yshift=-\padding] I-\the\numexpr\rowResult+\numRowsK-1\relax-\the\numexpr\colResult+\numColsK-1\relax.south east);
					\coordinate (Is-sw) at (Is-nw |- Is-se);
					\coordinate (Is-ne) at (Is-se |- Is-nw);
					
					\filldraw[red,   fill opacity=.1] (Is-nw) rectangle (Is-se);
					\filldraw[green, fill opacity=.1] (I*K-\rowResult-\colResult.north west) rectangle (I*K-\rowResult-\colResult.south east);
					
					\draw[blue, dotted] 
					(Is-nw) -- (K.north west)
					(Is-se) -- (K.south east)
					(Is-sw) -- (K.south west)
					(Is-ne) -- (K.north east)
					;
					\draw[green, dotted] 
					(I*K-\rowResult-\colResult.north west) -- (K.north west)
					(I*K-\rowResult-\colResult.south east) -- (K.south east)
					(I*K-\rowResult-\colResult.south west) -- (K.south west)
					(I*K-\rowResult-\colResult.north east) -- (K.north east)
					;
					
					\draw[blue,  fill=blue!10!white] (K.north west) rectangle (K.south east);
					
					\foreach \row [evaluate=\row as \rowI using int(\row+\rowResult-1)] in {1, ..., \numRowsK} {%
						\foreach \col [evaluate=\col as \colI using int(\col+\colResult-1)] in {1, ..., \numColsK} {%
							\node[text=blue] at (I-\rowI-\colI.south east) [xshift=-.3em] {\tiny$\times \K{\row}{\col}$};
						}
					}
				\end{scope}
				
				% ------- labels -------
				\tikzset{node distance=0em}
				\node[below=of I] (I-label) {$I$};
				\node at (K |- I-label)     {$K$};
				\node at (I*K |- I-label)   {$I \cross K$};
			\end{tikzpicture}
			\caption{Example of convolution of input, $I$, with the kernel, $K$. The kernel is the trainable weights.}
			\label{fig:convolution}
		\end{figure}
		
		The values of the kernel acting on the input are then summed and processed with some activation function yielding the final value of the convolution neuron. Mathematically we can write this convolution $G$ as a function of the input $I$ and kernel $K$ as,
		\begin{gather}\label{eq:cnn}
			G(I,K) = \sum_{j=1}^{3}\sum_{k=1}^{3} I(m-j, n-k) k(j,k) \nonumber \\
			= 1\cdot1+0\cdot0+0\cdot1+1\cdot0+1\cdot1+0\cdot0+1\cdot1+1\cdot0+1\cdot1 = 4
		\end{gather}
		where $j$ and $k$ are summing over the indices of the kernel, in this case ae 3x3 matrices, and $m$ and $n$ are the row and column indices of the input according to the notation of Figure \ref{fig:convolution}. Then one can apply an activation function to this result.
		
		Note that the kernel requires input from surrounding values of the input, one can not simply apply the kernel to edge values. Depending on the size of the kernel one would thus naturally get a size reduction in the convolution layer. It is however possible to apply some padding like zero-padding to pad around the edge such that the kernel can be applied around the edge such that the convolution layer retains the same size as the input.
		
		Another option in case one seeks to reduce the size of the convolution layer is to change the stride. The striding regulates the stepping which the kernel should move across the input. One can thus apply a stride of two to only compute a value for every second step. This will then reduce the size of the convolution layer by a factor of two.
		
		The final important thing to consider with convolution layers is the number of filters. Since you can generally reduce the number of parameters by orders of magnitudes with convolution layers, one might face an issue with information being lost as the convolution might condense the problem with information loss. To compensate one can have multiple filters in convolution layers, in this case, one essentially repeat the convolution process as described in Figure \ref{fig:convolution}. In this way, each point of the input is compiled multiple times with different weights that can be trained to capture different information about each point. For, example the filters might each capture some color value.
		
		Overall, the convolution layer helps to massively reduce the number of parameters, but one should consider additional hyperparameters such as filter number, kernel size, padding and stride.
		
		Another operation relevant to CNNs is that of pooling. In practice, this is a layer, but it doesn't add any new trainable parameters. It provides a systematical method to merge output values of the previous layer to decrease the number of parameters passed forward in the network as shown in Figure \ref{fig:maxpool} with max pooling.
		
		\begin{figure}
			\centering
			\begin{tikzpicture}
				\fill [c1] (0, 0) rectangle (2, 2);
				\fill [c2]   (2, 0) rectangle (4, 2);
				\fill [c3] (0, 2) rectangle (2, 4);
				\fill [c4]   (2, 2) rectangle (4, 4);
				
				\fill [c1]  (8, 1) rectangle (9, 2);
				\fill [c2]   (9, 1) rectangle (10, 2);
				\fill [c3]  (8, 2) rectangle (9, 3);
				\fill [c4]   (9, 2) rectangle (10, 3);
				
				\foreach \i in {\xMin,...,\xMax} {
					\draw [very thin,gray] (\i,\yMin) -- (\i,\yMax)  node [below] at (\i,\yMin) {};
				}
				\foreach \i in {\yMin,...,\yMax} {
					\draw [very thin,gray] (\xMin,\i) -- (\xMax,\i) node [left] at (\xMin,\i) {};
				}
				
				\foreach \i in {\xMin,2,...,\xMax} {
					\draw [thick,gray] (\i,\yMin) -- (\i,\yMax)  node [below] at (\i,\yMin) {};
				}
				\foreach \i in {\yMin,2,...,\yMax} {
					\draw [thick,gray] (\xMin,\i) -- (\xMax,\i) node [left] at (\xMin,\i) {};
				}
				\node at (0.5, 0.5) {7};
				\node at (1.5, 0.5) {9};
				\node at (2.5, 0.5) {3};
				\node at (3.5, 0.5) {5};
				%
				\node at (0.5, 1.5) {0};
				\node at (1.5, 1.5) {7};
				\node at (2.5, 1.5) {0};
				\node at (3.5, 1.5) {0};
				%
				\node at (0.5, 2.5) {5};
				\node at (1.5, 2.5) {0};
				\node at (2.5, 2.5) {9};
				\node at (3.5, 2.5) {3};
				%
				\node at (0.5, 3.5) {9};
				\node at (1.5, 3.5) {2};
				\node at (2.5, 3.5) {9};
				\node at (3.5, 3.5) {6};
				
				\draw[draw=black,line width=5pt,-{Latex[length=5mm]}] (4.5, 2)  -- (7.5,2);
				\node[font=\footnotesize\bfseries] at (6, 2.5) {$\mathbf{2\times 2}$ max pooling};
				
				\foreach \i in {\xMinR,...,\xMaxR} {
					\draw [thick,gray] (\i,\yMinR) -- (\i,\yMaxR)  node [below] at (\i,\yMinR) {};
				}
				\foreach \i in {\yMinR,...,\yMaxR} {
					\draw [thick,gray] (\xMinR,\i) -- (\xMaxR,\i) node [left] at (\xMinR,\i) {};
				}
				
				\node at (8.5, 1.5) {9};
				\node at (9.5, 1.5) {5};
				\node at (8.5, 2.5) {9};
				\node at (9.5, 2.5) {9};
				
				\draw [decorate,decoration={brace,amplitude=4pt},xshift=-2pt,yshift=0pt]
				(0,2) -- (0,4) node [black,midway,xshift=-0.3cm] {\footnotesize $2$};
				
				\draw [decorate,decoration={brace,amplitude=4pt},xshift=0pt,yshift=2pt]
				(0,4) -- (2,4) node [black,midway,yshift=+0.3cm] {\footnotesize $2$};
			\end{tikzpicture}
			\caption{Example of using max pooling with a pool size of two and a stride value of two.}
			\label{fig:maxpool}
		\end{figure}
		
		There are a few different ways to do the pooling such as min, max and average pooling. In the case of max pooling one reduces the pool size to a single value by selecting the maximum value in the given pool as shown in Figure \ref{fig:maxpool}. In this example the pool size is two, thus a $2\cross2$ pool is considered and the highest value is selected. Since the stride is of value two, then one skips one ahead before creating another pool. Had the stride been of value one, then there would be overlap between the pools, but it might be desirable to choose such a value. As with the convolution layer one can also add padding to a pooling layer. The pooling layers generally have pool size, stride and padding as hyperparameters.
		
		There is another ingredient you often need to consider when working with CNNs. Recall that you use convolution and pooling layer to handle a larger number of parameters and to condense them into an object with less parameters. In imagine processing you will often use these layer types to reduce the input until it is feasible to use dense layers. In case you have a multi dimensional object, you cannot just parse that to a dense layer as it expects a vector input. For this you can use a flatten function to map the multi dimension object layer to a vector which can then be feed to an ANN.
		
		\begin{figure}[t!]
			\centering
			\begin{tikzpicture}[thick,scale=0.85]
				\node at (0.5,-1){\begin{tabular}{c}input image\\layer $l = 0$\end{tabular}};
				
				\draw (0,0) -- (1,0) -- (1,1) -- (0,1) -- (0,0);
				
				\node at (3,3.5){\begin{tabular}{c}convolutional \\ layer $l = 1$\end{tabular}};
				
				\draw[fill=black,opacity=0.2,draw=black] (2.75,1.25) -- (3.75,1.25) -- (3.75,2.25) -- (2.75,2.25) -- (2.75,1.25);
				\draw[fill=black,opacity=0.2,draw=black] (2.5,1) -- (3.5,1) -- (3.5,2) -- (2.5,2) -- (2.5,1);
				\draw[fill=black,opacity=0.2,draw=black] (2.25,0.75) -- (3.25,0.75) -- (3.25,1.75) -- (2.25,1.75) -- (2.25,0.75);
				\draw[fill=black,opacity=0.2,draw=black] (2,0.5) -- (3,0.5) -- (3,1.5) -- (2,1.5) -- (2,0.5);
				\draw[fill=black,opacity=0.2,draw=black] (1.75,0.25) -- (2.75,0.25) -- (2.75,1.25) -- (1.75,1.25) -- (1.75,0.25);
				\draw[fill=black,opacity=0.2,draw=black] (1.5,0) -- (2.5,0) -- (2.5,1) -- (1.5,1) -- (1.5,0);
				
				\node at (4.5,-1){\begin{tabular}{c}pooling layer\\layer $l = 3$\end{tabular}};
				
				\draw[fill=black,opacity=0.2,draw=black] (5,1.25) -- (5.75,1.25) -- (5.75,2) -- (5,2) -- (5,1.25);
				\draw[fill=black,opacity=0.2,draw=black] (4.75,1) -- (5.5,1) -- (5.5,1.75) -- (4.75,1.75) -- (4.75,1);
				\draw[fill=black,opacity=0.2,draw=black] (4.5,0.75) -- (5.25,0.75) -- (5.25,1.5) -- (4.5,1.5) -- (4.5,0.75);
				\draw[fill=black,opacity=0.2,draw=black] (4.25,0.5) -- (5,0.5) -- (5,1.25) -- (4.25,1.25) -- (4.25,0.5);
				\draw[fill=black,opacity=0.2,draw=black] (4,0.25) -- (4.75,0.25) -- (4.75,1) -- (4,1) -- (4,0.25);
				\draw[fill=black,opacity=0.2,draw=black] (3.75,0) -- (4.5,0) -- (4.5,0.75) -- (3.75,0.75) -- (3.75,0);
				
				\node at (7,3.5){\begin{tabular}{c}convolutional \\ layer $l = 4$\end{tabular}};
				
				\draw[fill=black,opacity=0.2,draw=black] (7.5,1.75) -- (8.25,1.75) -- (8.25,2.5) -- (7.5,2.5) -- (7.5,1.75);
				\draw[fill=black,opacity=0.2,draw=black] (7.25,1.5) -- (8,1.5) -- (8,2.25) -- (7.25,2.25) -- (7.25,1.5);
				\draw[fill=black,opacity=0.2,draw=black] (7,1.25) -- (7.75,1.25) -- (7.75,2) -- (7,2) -- (7,1.25);
				\draw[fill=black,opacity=0.2,draw=black] (6.75,1) -- (7.5,1) -- (7.5,1.75) -- (6.75,1.75) -- (6.75,1);
				\draw[fill=black,opacity=0.2,draw=black] (6.5,0.75) -- (7.25,0.75) -- (7.25,1.5) -- (6.5,1.5) -- (6.5,0.75);
				\draw[fill=black,opacity=0.2,draw=black] (6.25,0.5) -- (7,0.5) -- (7,1.25) -- (6.25,1.25) -- (6.25,0.5);
				\draw[fill=black,opacity=0.2,draw=black] (6,0.25) -- (6.75,0.25) -- (6.75,1) -- (6,1) -- (6,0.25);
				\draw[fill=black,opacity=0.2,draw=black] (5.75,0) -- (6.5,0) -- (6.5,0.75) -- (5.75,0.75) -- (5.75,0);
				
				\node at (9.5,-1){\begin{tabular}{c}pooling layer\\layer $l = 6$\end{tabular}};
				
				\draw[fill=black,opacity=0.2,draw=black] (10,1.75) -- (10.5,1.75) -- (10.5,2.25) -- (10,2.25) -- (10,1.75);
				\draw[fill=black,opacity=0.2,draw=black] (9.75,1.5) -- (10.25,1.5) -- (10.25,2) -- (9.75,2) -- (9.75,1.5);
				\draw[fill=black,opacity=0.2,draw=black] (9.5,1.25) -- (10,1.25) -- (10,1.75) -- (9.5,1.75) -- (9.5,1.25);
				\draw[fill=black,opacity=0.2,draw=black] (9.25,1) -- (9.75,1) -- (9.75,1.5) -- (9.25,1.5) -- (9.25,1);
				\draw[fill=black,opacity=0.2,draw=black] (9,0.75) -- (9.5,0.75) -- (9.5,1.25) -- (9,1.25) -- (9,0.75);
				\draw[fill=black,opacity=0.2,draw=black] (8.75,0.5) -- (9.25,0.5) -- (9.25,1) -- (8.75,1) -- (8.75,0.5);
				\draw[fill=black,opacity=0.2,draw=black] (8.5,0.25) -- (9,0.25) -- (9,0.75) -- (8.5,0.75) -- (8.5,0.25);
				\draw[fill=black,opacity=0.2,draw=black] (8.25,0) -- (8.75,0) -- (8.75,0.5) -- (8.25,0.5) -- (8.25,0);
				
				\node at (12,3.5){\begin{tabular}{c}dense layer\\layer $l = 7$\end{tabular}};
				
				\draw[fill=black,draw=black,opacity=0.5] (10.5,0) -- (11,0) -- (12.5,1.75) -- (12,1.75) -- (10.5,0);
				
				\node at (13,-1){\begin{tabular}{c}dense output \\ layer $l = 8$\end{tabular}};
				
				\draw[fill=black,draw=black,opacity=0.5] (12.5,0.5) -- (13,0.5) -- (13.65,1.25) -- (13.15,1.25) -- (12.5,0.5);
			\end{tikzpicture}
			\caption{The architecture of the original convolutional neural network, as introduced by LeCun et al. (1989), alternates between convolutional layers and pooling layers. The feature maps of the final polling layer are then fed into the actual classifier consisting of an arbitrary number of fully connected layers. The output layer usually uses softmax activation functions.}
			\label{fig:cnn}
		\end{figure}
	
		As a result a classical CNN will look like Figure \ref{fig:cnn}, with layers of a convolutional layer followed by a pooling layer. Then the there is a flattening step before the input is feed into densely connected layers. In this case we can imagine a network with a softmax function in the output layer to classify the input into a fixed amount of categories. The combination of many types of layers suggests that we are now entering the realm of DP methods.
		
		To understand a CNN a bit better, let us discuss the effect of the convolution layer. For each submatrix of the full input we are scanning for features. If we think about animals or humans, this could be features like eyes. Now there could be multiple features in one submatrix, and therefore you usually have multiple feature maps per submatrix. In other words the procedure described in Equation \ref{eq:cnn} is done multiple times and the associated with different weights, one for each feature map. In the case of RGB pictures it is also common practice to have a layer for each color, thus in this case three. So lets say you have and RGB picture and you want to extract 4 features, you would then need 12 feature maps or filters as they are also called.
		
		As you are generally reducing the input seize with convolution and poolings layers, you are naturally increasing the scale of the feature detection of the convolution layer, because you kernel is the same size and your layer input is containing information about a larger area. That creates a hierarchy of how the learning is done. In the first layers the network captures small features and then the subsequent layers capture larger features.
		
		Regarding compute, there one should consider using GPU resources when using of networks with CNN features. The computations related to convolution and pooling layers are very simple, but involve a lot of computations. This allows for effective parallelization with modern GPUs, as they have an abundance of cores at the order of $10^3$ compared to CPUs with core numbers typically at the order of $10^1$. In particular modern GPUs tensor cores are highly efficient at these types of workloads. 
		
	\subsection{Recurrent neural networks}
		In certain applications such as speech recognition, and other cases with time dependence, the previous input value may be correlated with the subsequent value. This is something a usual DNN or CNN cannot take into account at the neuron level. This requires the neuron to have some memory. To achieve this one can make use of recurrent neurons to make recurrent neural networks (RNN) \cite{schmidt:2019}.
		
		The simplest way to incorporate memory is to modify the neuron to store the previous output and include it in the current input. In this way, the historical output is used for the current evaluation in the neuron. This concept of recurrent neurons is illustrated in Figure \ref{fig:recurrent-node}. Mathematically the math in the neuron changes in the following way,
		\begin{gather}
			a(z_j(x_i)) = f(w_{ij, 1}x_i + w_{ij, 2}h_{i-1} + b_j \cdot \vec{1}),
		\end{gather}
		where $h_{i-1}$ is the hidden state associated with the previous output of the neuron. The hidden state generally captures an encoding of the most recent neuron computation. This hidden state is a new object in recurrent neurons and it allows for short term memory in the neurons. Since both $x_1$ and $h_i$ are inputs to the function, they both get a weight matrix associated to them.
		\begin{figure}
			\centering
			\begin{tikzpicture}[item/.style={circle,draw,thick,align=center},
				itemc/.style={item,on chain,join}]
				\begin{scope}[start chain=going right,nodes=itemc,every
					join/.style={-latex,very thick},local bounding box=chain]
					\path node (A0) {$A$} node (A1) {$A$} node (A2) {$A$} node[xshift=2em] (At)
					{$A$};
				\end{scope}
				\node[left=1em of chain,scale=2] (eq) {$=$};
				\node[left=2em of eq,item] (AL) {$A$};
				\path (AL.west) ++ (-1em,2em) coordinate (aux);
				\draw[very thick,-latex,rounded corners] (AL.east) -| ++ (1em,2em) -- (aux) 
				|- (AL.west);
				\foreach \X in {0,1,2,t} 
				{\draw[very thick,-latex] (A\X.north) -- ++ (0,2em)
					node[above,item,fill=red!10] (h\X) {$h_\X$};
					\draw[very thick,latex-] (A\X.south) -- ++ (0,-2em)
					node[below,item,fill=green!10] (x\X) {$x_\X$};}
				\draw[white,line width=0.8ex] (AL.north) -- ++ (0,1.9em);
				\draw[very thick,-latex] (AL.north) -- ++ (0,2em)
				node[above,item,fill=red!10] {$h_t$};
				\draw[very thick,latex-] (AL.south) -- ++ (0,-2em)
				node[below,item,fill=green!10] {$x_t$};
				\path (x2) -- (xt) node[midway,scale=2,font=\bfseries] {\dots};
			\end{tikzpicture}
			\caption{Illustration of recurrent neurons using the previous output as input for the current node computation.}
			\label{fig:recurrent-node}
		\end{figure}
		
		One of the limitations of the simple recurrent neuron is that it only takes into account the previous output. This means we are limited to short-term memory. Using more complex logic it is possible to achieve long-term memory, and for this one can use long short-term memory (LSTM) layers. These layers a significantly more complex and compute-intensive, but the long term cells in these layers are capable of storing information long-term. Such models are useful in the case where there are long-term memory is needed to obtain the correct prediction.
		
		A LSTM layer is a layer of LSTM cells, with each cell containing the more complex logic as shown in Figure \ref{fig:lstm}. To understand this cell we need to consider the different inputs it takes and the outputs it yields:
		\begin{itemize}
			\item $x_i$: The current input to the cell
			\item $h_{i-1}$: The hidden state input to the cell
			\item $c_{i-1}$: The cell state input to the cell
			\item $h_{i}$: The hidden state output of the cell
			\item $c_{i}$: The cell state output of the cell
		\end{itemize}
		Compared to the simple recurrent neuron, the LSTM cell contains a new object which is the cell state. This is the object that holds long term memory that forms across many inputs unlike the hidden state which contains information related to the previous input. It is the updating and handling of the cell state that makes the LSTM cell much more complex due to the logic behind updating and using it.
		
		\begin{figure}
			\centering
			\begin{tikzpicture}[
				% GLOBAL CFG
				font=\sf \scriptsize,
				>=LaTeX,
				% Styles
				cell/.style={% For the main box
					rectangle, 
					rounded corners=5mm, 
					draw,
					very thick,
				},
				operator/.style={%For operators like +  and  x
					circle,
					draw,
					inner sep=-0.5pt,
					minimum height =.2cm,
				},
				function/.style={%For functions
					ellipse,
					draw,
					inner sep=1pt
				},
				ct/.style={% For external inputs and outputs
					circle,
					draw,
					line width = .75pt,
					minimum width=1cm,
					inner sep=1pt,
				},
				gt/.style={% For internal inputs
					rectangle,
					draw,
					minimum width=4mm,
					minimum height=3mm,
					inner sep=1pt
				},
				mylabel/.style={% something new that I have learned
					font=\scriptsize\sffamily
				},
				ArrowC1/.style={% Arrows with rounded corners
					rounded corners=.25cm,
					thick,
				},
				ArrowC2/.style={% Arrows with big rounded corners
					rounded corners=.5cm,
					thick,
				},
				]
				
				%Start drawing the thing...    
				% Draw the cell: 
				\node [cell, minimum height =4cm, minimum width=6cm] at (0,0){} ;
				
				% Draw inputs named ibox#
				\node [gt] (ibox1) at (-2,-0.75) {$\sigma$};
				\node [gt] (ibox2) at (-1.5,-0.75) {$\sigma$};
				\node [gt, minimum width=1cm] (ibox3) at (-0.5,-0.75) {Tanh};
				\node [gt] (ibox4) at (0.5,-0.75) {$\sigma$};
				
				% Draw opérators   named mux# , add# and func#
				\node [operator] (mux1) at (-2,1.5) {$\times$};
				\node [operator] (add1) at (-0.5,1.5) {+};
				\node [operator] (mux2) at (-0.5,0) {$\times$};
				\node [operator] (mux3) at (1.5,0) {$\times$};
				\node [function] (func1) at (1.5,0.75) {Tanh};
				
				% Draw External inputs? named as basis c,h,x
				\node[ct, label={[mylabel]Cell}] (c) at (-4,1.5) {$c_{i-1}$};
				\node[ct, label={[mylabel]Hidden}] (h) at (-4,-1.5) {$h_{i-1}$};
				\node[ct, label={[mylabel]left:Input}] (x) at (-2.5,-3) {$x_{i}$};
				
				% Draw External outputs? named as basis c2,h2,x2
				\node[ct, label={[mylabel]Label1}] (c2) at (4,1.5) {$c_{i}$};
				\node[ct, label={[mylabel]Label2}] (h2) at (4,-1.5) {$h_{i}$};
				\node[ct, label={[mylabel]left:Label3}] (x2) at (2.5,3) {$h_{i}$};
				
				% Start connecting all.
				%Intersections and displacements are used. 
				% Drawing arrows    
				\draw [ArrowC1] (c) -- (mux1) -- (add1) -- (c2);
				
				% Inputs
				\draw [ArrowC2] (h) -| (ibox4);
				\draw [ArrowC1] (h -| ibox1)++(-0.5,0) -| (ibox1); 
				\draw [ArrowC1] (h -| ibox2)++(-0.5,0) -| (ibox2);
				\draw [ArrowC1] (h -| ibox3)++(-0.5,0) -| (ibox3);
				\draw [ArrowC1] (x) -- (x |- h)-| (ibox3);
				
				% Internal
				\draw [->, ArrowC2] (ibox1) -- (mux1);
				\draw [->, ArrowC2] (ibox2) |- (mux2);
				\draw [->, ArrowC2] (ibox3) -- (mux2);
				\draw [->, ArrowC2] (ibox4) |- (mux3);
				\draw [->, ArrowC2] (mux2) -- (add1);
				\draw [->, ArrowC1] (add1 -| func1)++(-0.5,0) -| (func1);
				\draw [->, ArrowC2] (func1) -- (mux3);
				
				%Outputs
				\draw [-, ArrowC2] (mux3) |- (h2);
				\draw (c2 -| x2) ++(0,-0.1) coordinate (i1);
				\draw [-, ArrowC2] (h2 -| x2)++(-0.5,0) -| (i1);
				\draw [-, ArrowC2] (i1)++(0,0.2) -- (x2);
				
			\end{tikzpicture}
			\caption{A diagram of the LSTM cell with an input, a hidden state, and a cell state with internal logic to update the states.}
			\label{fig:lstm}
		\end{figure}
		
		There are three main logical gates computed in the LSTM cell. The first to consider it he forget gate $f_{\rm{forget}}$. This gate considers the input $x_i$ and the hidden state $h_{i-1}$. Here the computation is as follows,
		\begin{gather}
			f_{\rm{forget}}(x_i, h_{i-1}) = \sigma \paa{w_{ij, 1}x_i + w_{ij, 2}h_{i-1} + b_j \cdot \vec{1}},
		\end{gather}
		which is then multiplied to the cell state $c_{i-1}$ to get a scaled cell state $c_{i-1}'$,
		\begin{gather}
				c_{i-1}' = f_{\rm{forget}}(x_i, h_{i-1}) c_{i-1}.
		\end{gather}
		Since $0 \le f_{\rm{forget}} \le 1$ the result of the forget gate is to decide how much of the past should be forgotten, with $ f_{\rm{forget}} \sim 1$ yielding a minimal change of the cell state $c_{i-1}' \sim c_{i-1}$. On the other hand $ f_{\rm{forget}} \ll 1$ yields $c_{i-1}' \sim  0$, thus wiping the long term memory clean.
		
		The next gate is the input gate $f_{\rm{input}}$ which computes a quantity like $f_{\rm{forget}}$ and multiplies it with a similar function where the tanh function is used,
		\begin{gather}
			f_{\rm{input}} =\sigma \paa{w_{ij, 3}x_i + w_{ij, 4}h_{i-1} + b_j \cdot \vec{1}} \cdot \tanh \paa{w_{ij, 5}x_i + w_{ij, 6}h_{i-1} + b_j \cdot \vec{1}},
		\end{gather}
		where one can either do a point or piece wise multiplication of the two objects. The part with the sigmoid function can be considered as a scaling term like in the forget gate, and the part with the tanh function is the normalized encoding of the input plus the hidden state. The cell state is then updated,
		\begin{gather}
			c_{i} = c_{i-1}' + f_{\rm{input}},
		\end{gather}
		and that cell state is then used in the next input step and to compute $h_i$ in the last gate, the output gate $f_{\rm{output}}$. This gate has the same components as the previous gate,
		\begin{gather}
			f_{\rm{output}} =\sigma \paa{w_{ij, 7}x_i + w_{ij, 8}h_{i-1} + b_j \cdot \vec{1}} \cdot \tanh \paa{w_{ij, 9}c_i + b_j \cdot \vec{1}} = h_i,
		\end{gather}
		and the result is the new hidden state which is also the output of the LSTM cell. Note that the output gate contains both the input, hidden state and the updated cell state.
		
		In the LSTM cell we have 9 weight matrices and a lot of biases to train. Likewise we have a lot of logic to process. This means that these cells are computationally much more intense compared to a simple neuron. Due to the extra complexity, it is often favorable to use CPU cores to deal with LSTM networks as their more powerful cores outclass the parallelization which CPUs offer.
	
	\subsection{Reservoir computing}
		If you have no clue what a suitable architecture is and you need memory, you might want to try out reservoir computing. This is an actively evolving field where we take even more inspiration from biology, in particular the human brain.
		
		The concept is as follows, you take a bunch of neurons, which could be recurrent neurons with memory, and then connect them sparse. The result is hundreds or thousands of neurons connected randomly to each other, not densely and not in a particular way. This is similar to how neurons in our brains are connected. Human neurons only connect to a handful of neurons in the vicinity. The same is idea of reservoir computing, this construction of randomly connected neurons is our reservoir. On its own, this cannot achieve anything, you just have a $N$ neurons. 
		
		Next consider some input signal $u(t) \in \mathbb{R}^{N_u}$ which we connect to $N_u$ neurons in the reservoir with weights $W^{\rm{in}}$. The reservoir's reservoir's high-dimensional are given as $X(t) \in \mathbb{R}^N$ and randomly initialized weight $W$, and in this reservoir there are states accessible for read out $x(t) \in \mathbb{R}^{N_x}$ where $N_u + N_x \ll N$ with wights $W^{\rm{out}}$. 
		
		The prediction $\hat{y}$ can then be defined as,
		\begin{gather}
			\hat{y} = g \paa{W^{\rm{out}} x(t)},\\
			x(t) = f\paa{W x(t) + W^{\rm{in}} u(t)} ,
		\end{gather}
		with $f$ and $g$ being two activation functions.
		
		It is possible to then train the network in a supervised way, or in other ways. The key to this method is that one does not train the reservoir, thus $W$ is fixed, but on can train $W^{\rm{in}}$ and $W^{\rm{out}}$ for inputs $u$ and true outputs $y$. We are thus optimizing how we write and read to the reservoir. It is also possible to have ANN's connected before and after the reservoir, and that ANN may be fully trainable too. The main idea is simply that we do not care about optimizing the reservoir itself, but rather our interaction with it. It is possible to try to train the reservoir itself, with on of the more straight forward methods would be using genetic learning.
	
	\subsection{Exercises}
		In this week the exercises allow you get practical experience with different types of networks using examples from \texttt{tensorflow}. We look doing image classification of using ANNs and CNNs to compare the practicality of the different methods. The first two exercises use the CIFAR10 dataset containing 60,000 color images in 10 classes, with 6,000 images in each class. The dataset is divided into 50,000 training images and 10,000 testing images. The classes are mutually exclusive and there is no overlap between them. Then we consider text classification with memory. Be careful with overfitting! These example should give you a practical feel for the practicality of different methods in different situations. The code for this week and be found in the folder \texttt{week 3}.\newline
		
		\textbf{Exercise 1: ANN image classification}\newline
		In this exercise you need to construct a ANN that can correctly classify the image specified in the script \texttt{ann.py}. Note you need to flatten the input first before you can use dense layers. How many parameters to you need to obtain an accuracy of 30\%, 40\% or 50\%?\newline
		
		\textbf{Exercise 2: CNN image classification}\newline
		In this exercise you need to construct a CNN that can correctly classify the image specified in the script \texttt{cnn.py}. How many parameters to you need to obtain an accuracy of 60\%, 70\% or 80\%? Compare the results to the ANN network. How is the performance and how do the parameter number differ? \newline
		
		\textbf{Exercise 3: Text classification}\newline
		In this exercise you need to classify text. This example used IMDB data and the goal is to determine if a text represent a negative or positive review. This is called sentiment analysis. Because we are working with text, we again need to consider encoding and also tokenization. The latter concept is beyond this lesson and you may study it if you are interested. The code for this exercise is in \texttt{text.py} and you need to construct a network with memory. To get started you are presented with the initial encoding and embedding layer and then it is up to you to implement layers such as bidirectional, recurrent neuron, and LSTM layers. To make RNN layers you can use subclassing, check the \texttt{keras} or \texttt{tensorflow} documentation on how to do it. Note LSTM layers are very compute heavy and you generally don't need a lot of them. The code may produce errors that have no functional meaning, they relate to the import method of the dataset. Can yo reach 90\% accuracy?
		
		
% -------------------------------------------------------------------
% Bibliography
% -------------------------------------------------------------------
\bibliographystyle{plain} % We choose the "plain" reference style
\bibliography{lecture_notes} % Entries are in the refs.bib file

\end{document}
